\documentclass[12pt,a4paper]{article}
\usepackage{amsmath}
\usepackage{amsthm}
\usepackage{amsfonts}
\usepackage{amssymb}
\usepackage{amsmath,amscd}
\usepackage[symbol]{footmisc}
\usepackage{fancyhdr}
\usepackage{graphicx}
\usepackage{bm}
\usepackage[english]{babel}
\linespread{1.25}


\DeclareMathOperator{\Tr}{Tr}
\newcommand{\eit}{e^{i \theta}}
\newcommand{\eits}{e^{2i \theta}}
\newcommand{\emit}{e^{-i \theta}}
\newcommand{\emits}{e^{-2i \theta}}
\DeclareMathOperator{\D}{D}
\DeclareMathOperator{\Real}{Re}
\DeclareMathOperator{\T}{\text{\scriptsize T}}

\renewcommand{\thefootnote}{\fnsymbol{footnote}}


\begin{document}
\section{Gaussian model for the $(2,0)$ Dirac operator}
\subsection{Notation}
\begin{equation}
D = \gamma_1 \otimes \{H_1, \cdot \} + \gamma_2 \otimes \{H_2, \cdot \}.
\end{equation}
Define $\gamma_{\pm} := \frac{1}{2}(\gamma_1 \mp i \gamma_2)$ and $W := H_1 + i H_2$. $D$ becomes:
\begin{equation}
D = \gamma_+ \otimes \{ W, \cdot \} + \gamma_- \otimes \{ W^\dagger, \cdot \}
\end{equation}
And the following relations hold:
\begin{align}
\gamma_\pm^2 &= 0 \label{prop1} \\
\gamma_\pm \gamma_\mp &= \frac{1}{2}(I_C \pm i \gamma_1 \gamma_2) \label{prop2} \\
(\gamma_\pm \gamma_\mp)^2 &= \gamma_\pm \gamma_\mp \label{prop3}
\end{align}
where $I_C$ is the identity in the space of $\gamma$ matrices.
\subsection{Action, $\Tr D^2$ part}
Because of Eq.(\ref{prop1}), only the cross terms survive in $D^2$:
\begin{equation}
D^2 = \gamma_+ \gamma_- \otimes \{ W, \cdot \} \{ W^\dagger , \cdot \} + \gamma_- \gamma_+ \otimes \{ W^\dagger , \cdot \} \{ W , \cdot \}
\end{equation}
Taking the trace yields:
\begin{equation}
\Tr D^2 = 2 C (n \Tr W W^{\dagger} + \Tr W \Tr W^{\dagger})
\end{equation}
where $C$ is the dimension of the $\gamma$ matrices and $n$ the dimension of the $W$ matrix.
\subsection{Action, $\Tr D^4$ part}
Because of Eq.(\ref{prop1}), the cross terms vanish when squaring $D^2$:
\begin{equation}
D^4 =  \gamma_+ \gamma_- \otimes (\{ W, \cdot \} \{ W^\dagger , \cdot \})^2 + \gamma_- \gamma_+ \otimes (\{ W^\dagger , \cdot \} \{ W , \cdot \})^2
\end{equation}
Taking the trace yields:
\begin{align}
\Tr D^4 &= 2 C \big[ n \Tr(W W^\dagger)^2 + \Tr W^2 \Tr W^{\dagger 2} + 2 (\Tr W W^\dagger)^2 \notag \\ 
&+ 2 \Tr W^2 W^\dagger \Tr W^\dagger + 2 \Tr W^{\dagger 2} W \Tr W \big].
\end{align}
\subsection{Stationary point}
Stationary points are solutions to the equation
\begin{equation}
\delta S = S[W + \delta W] - S[W] = 0 \quad  \forall \ \delta W \ll 1.
\end{equation}
To first order in $\delta W$ the terms $\Tr D^2$ and $\Tr D^4$ give:
\begin{equation}
\delta \Tr D^2 = 2 C \big[ n \Tr W \delta W^\dagger + n \Tr W^\dagger \delta W + \Tr W \Tr \delta W^\dagger + \Tr W^\dagger \Tr \delta W \big]
\end{equation}
\begin{align}
\delta \Tr D^4 &= 2 C \big[ 2n \Tr ( W W^\dagger W \delta W^\dagger) + 2n \Tr ( W^\dagger W W^\dagger \delta W) \notag \\ 
&+ 2 \Tr W^2 \Tr W^\dagger \delta W^\dagger +  2 \Tr W^{\dagger 2} \Tr W \delta W \notag \\
&+ 4 \Tr W W^\dagger ( \Tr W \delta W^\dagger + \Tr W^\dagger \delta W) \notag \\ 
&+ 2 \Tr W^2 W^\dagger \Tr \delta W^\dagger + 2 \Tr W^{\dagger 2} W \Tr \delta W \notag \\
&+ 2 \Tr W^\dagger ( \Tr W^2 \delta W^\dagger + \Tr W W^\dagger \delta W + \Tr W^\dagger W \delta W) \notag \\
&+ 2 \Tr W ( \Tr W^{\dagger 2} \delta W + \Tr W^\dagger W \delta W^\dagger + \Tr W W^\dagger \delta W^\dagger) \big].
\end{align}
We check for solutions of the form:
\begin{equation}
W = I \rho e^{i \theta}
\end{equation}
with $\rho > 0$ and $\theta \in [0, 2 \pi )$.
\begin{align}
\Tr D^2&: \quad 4 C n \rho \ ( \eit \Tr \delta W^\dagger + \emit \Tr \delta W) \\
\Tr D^4&: \quad 4 C n \rho \ ( 8 \rho^2 \eit \Tr \delta W^\dagger + 8 \rho^2 \emit \Tr \delta W)
\end{align}
putting the coefficient of $\delta W$ or $\delta W^\dagger$ to zero gives:
\begin{equation}
\rho^2 = 0 \quad \text{or} \quad \rho^2 = -\frac{g}{8}.
\end{equation}
\subsection{Quadratic action in $\delta W$}
Write $W = \eit(\rho + \epsilon V)$ for small $\epsilon$ and traceless $V$. We will separate the contributions to the action in orders of $\epsilon$.
\begin{align}
\Tr D^2 \notag \\
O(1)&: \quad 4Cn^2 \rho^2 \\
O(\epsilon)&: \quad 4Cn \rho ( \Tr V^\dagger + \Tr V) \\
O(\epsilon^2)&: \quad 2C (n \Tr V V^\dagger + \Tr V \Tr V^\dagger)
\end{align}
\begin{align}
\Tr D^4 \notag \\
O(1)&: \quad 16Cn^2 \rho^4 \\
O(\epsilon)&: \quad 24 C n \rho^3 \Tr V + \text{c.c.} \\
O(\epsilon^2)&: \quad 8C\rho^2(2n \Tr V V^\dagger + n \Tr V V \\
&\phantom{:\quad \ }+2 \Tr V \Tr V^\dagger + \Tr V \Tr V + \text{c.c.}) \notag 
\end{align}
Since $V$ is traceless, the action to second order in $\epsilon$ reads:
\begin{align}
S &= 2Cn^2(g \rho^2 + 4\rho^4) + Cn \epsilon^2 \Tr V V^\dagger (g + 16 \rho^2) \notag \\
&\phantom{\quad}+ 8Cn \epsilon^2 \rho^2 \Tr V V + \text{c.c.}
\end{align}
Or:
\begin{equation}
S = 4Cn^2(g \rho^2 + 4 \rho^4) + Cn \epsilon^2 \Tr(\vec{V}^T M \vec{V}) 
\end{equation}
with $\vec{V}^T = (V, V^\dagger)$ and:
\begin{equation}
M = \begin{bmatrix}
    8 \rho^2 & g + 16 \rho^2 \\
    g + 16 \rho^2 & 8 \rho^2 \\
\end{bmatrix}
\end{equation}

\end{document}
